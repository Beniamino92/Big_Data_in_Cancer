\usepackage{setspace}
\usepackage{multirow}
%\usepackage{subfig}
\usepackage{amsmath,amssymb,amsfonts,dsfont}
\usepackage{natbib}
\usepackage{fullpage}
\usepackage{parskip} % stop indenting paragraphs
\usepackage{changepage} % allow indenting of whole sections of text
\usepackage{array}

\usepackage{enumitem} % stop indenting bullet points in itemized lists, fix spacing between list items
    \setlist[itemize]{itemsep=20pt}
    \setlist[enumerate]{itemsep=40pt}
    \setlist[enumerate,1]{itemsep=40pt}
    \setlist[description]{itemsep=20pt}

% draw decision trees
\usepackage{pgf}
\usepackage{tikz}

\usepackage{graphicx}
\usepackage[top=1in, bottom=1in, left=1in, right=1in]{geometry}
\usepackage{epstopdf}
\usepackage{color}                                          % allow use of colour in R code (for example)
\usepackage[none]{hyphenat}                        % don't hyphenate words
\usepackage{hyperref}                                    % allows hyperlinking of references

\usepackage{tcolorbox}
\usepackage{caption}
     \captionsetup{singlelinecheck=false}          % stops centring captions if they fit on one line
     \captionsetup{format=hang,margin=1in}    % indents caption to same width as table/fig with leftskip on it

\usepackage{subcaption}
\usepackage{csvsimple}                                   % allows autotabular function to import data from an R-generated csv

% define style for R code
\usepackage{listings} 
\lstset{language = R,
	basicstyle = \ttfamily\footnotesize,
           commentstyle = \color{blue}\textit,
           tabsize = 4,
           breakatwhitespace = true,
	showspaces = false,
           xleftmargin = .75cm,
	extendedchars = true}
          

% \doublespacing

\newtheorem{theorem}{Theorem}[section]
\newtheorem{assumption}{Assumption}
\newtheorem{prop}{Proposition}[section]
\newtheorem{lemma}{Lemma}[section]
\newtheorem{corollary}{Corollary}[section]
\newtheorem{conjecture}{Conjecture}[section]
\newtheorem{definition}{Definition}[section]
\newtheorem{remark}{Remark}[section] 

\DeclareMathOperator*{\argmin}{arg\,min}